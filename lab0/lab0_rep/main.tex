%!TeX program = xelatex
\documentclass{SYSUReport}
\usepackage{tabularx} % 在导言区添加此行
\usepackage{xcolor} % 已加载,无需重复
\usepackage{graphicx}
\lstset{
    basicstyle=\ttfamily\small,
    keywordstyle=\color{blue},
    commentstyle=\color{orange},
    stringstyle=\color{red},
    frame=single,
    breaklines=true,
    numbers=left,
    numberstyle=\tiny\color{gray}
}

% 根据个人情况修改
\headl{}
\headc{}
\headr{并行程序设计与算法实验}
\lessonTitle{并行程序设计与算法实验}
\reportTitle{Lab0-环境设置与串行矩阵乘法}
\stuname{李源卿}
\stuid{22336128}
\inst{计算机学院}
\major{计算机科学与技术}
\date{\today}

\begin{document}

% =============================================
% Part 1: 封面
% =============================================
\cover
\thispagestyle{empty} % 首页不显示页码
\clearpage

% % =============================================
% % Part 4: 正文内容
% % =============================================
% % 重置页码,并使用阿拉伯数字
% \pagenumbering{arabic}
% \setcounter{page}{1}

%%可选择这里也放一个标题
%\begin{center}
%    \title{ \Huge \textbf{{标题}}}
%\end{center}

\section{实验目的}
\begin{itemize}
   \item 理解并行程序设计的基本概念与理论。
    \item 掌握使用并行编程模型实现常见算法的能力。
    \item 学习评估并行程序性能的指标及其优化方法。
\end{itemize}

\section{实验内容}
\begin{itemize}
    \item 设计并实现以下矩阵乘法版本:
    \begin{itemize}
        \item 使用C/C++语言实现一个串行矩阵乘法。
        \item 比较不同编译选项、实现方式、算法或库对性能的影响:
        \begin{itemize}
            \item 使用Python实现的矩阵乘法。
            \item 使用C/C++实现的基本矩阵乘法。
            \item 调整循环顺序优化矩阵乘法。
            \item 应用编译优化提高性能。
            \item 使用循环展开技术优化矩阵乘法。
            \item 使用Intel MKL库进行矩阵乘法运算。
        \end{itemize}
    \end{itemize}
    \item 生成随机矩阵A和B,进行矩阵乘法运算得到矩阵C。
    \item 衡量各版本的运行时间、加速比、浮点性能等。
    \item 分析不同实现版本对性能的影响。
\end{itemize}
\section{实验思路以及实现}
\begin{itemize}
    \item 数据初始化
    生成的均是在[0,1]服从均匀分布的浮点数:
    \begin{lstlisting}[language=c]
// c/c++
#include <random>
random_device rd;  // 随机种子
mt19937 gen(rd()); // 随机数引擎
uniform_real_distribution<> dis(0.0, 1.0); // 均匀分布 [0, 1)
// 初始化矩阵
void initi(vector<vector<double>>&matrix,int rows,int cols){
    for (int i = 0; i < rows; ++i) {
        vector<double> row(cols);
        for (int j = 0; j < cols; ++j) {
            row[j] = dis(gen); 
        }
        matrix.push_back(row);
    }
    \end{lstlisting}
    \begin{lstlisting}[language=Python]
# Python
import random
def generate_random_matrix(rows, cols):
    """
    生成随机矩阵
    """
    return [[random.random() for _ in range(cols)] for _ in range(rows)]
    \end{lstlisting}

    \item 计时器

    \begin{lstlisting}[language=c]
//c语言
#include <sys/time.h>
gettimeofday(&start, NULL); // 开始计时
cblas_dgemm(CblasRowMajor, CblasNoTrans, CblasNoTrans,
            m, n, k, alpha, A, k, B, n, beta, C, n);
gettimeofday(&end, NULL); // 结束计时
    \end{lstlisting}
    \begin{lstlisting}[language=c++]
//c++
#include <chrono>
using namespace std::chrono;
auto start = high_resolution_clock::now();
cblas_dgemm(CblasRowMajor, CblasNoTrans, CblasNoTrans,
            m, n, k, alpha, A, k, B, n, beta, C, n);
auto end = high_resolution_clock::now();
    \end{lstlisting}
\begin{lstlisting}[language=Python]
#Python
import time
start=time.time()
C = matrix_multiply(A, B)
end=time.time()
    \end{lstlisting}
为了让MKL中的矩阵也初始化为[0,1]之间的数,我稍微修改了老师给的代码:
\begin{lstlisting}[language=c]
//MKL
for (i = 0; i < (m * k); i++) {
    A[i] = (double)rand() / RAND_MAX;
}

for (i = 0; i < (k * n); i++) {
    B[i] = (double)rand() / RAND_MAX;
}

for (i = 0; i < (m * n); i++) {
    C[i] = 0.0;
}
\end{lstlisting}
    \item 使用Python实现的矩阵乘法:
    
    \begin{lstlisting}[language=Python]
# Python矩阵乘法(仅给出关键部分)
for i in range(rows_A):          
        for j in range(cols_B):      
            for k in range(cols_A):  
                result[i][j] += A[i][k] * B[k][j]
    return result

    \end{lstlisting}
    \item 使用Numpy实现的矩阵乘法:\\Numpy中的矩阵乘法底层是通过c/c++实现的,效率较高。
\begin{lstlisting}[language=Python]
# Python矩阵乘法(仅给出关键部分)
import numpy as np
import time
m, k, n = 1000,1000 , 1000  
A = np.random.rand(m, k)    
B = np.random.rand(k, n)
start=time.time()
C = np.dot(A, B)
end=time.time()
print("time cost:",end-start)
    \end{lstlisting}
    \item 使用c++实现的普通矩阵乘法:
\begin{lstlisting}[language=c++]
//c++实现普通矩阵乘法
vector<vector<double>> matrixMultiply(const vector<vector<double>>& A, const vector<vector<double>>& B) {
    int rows_A = A.size();
    int cols_A = A[0].size();
    int rows_B = B.size();
    int cols_B = B[0].size();
    // 检查矩阵维度是否可乘
    if (cols_A != rows_B) {
        cerr << "Error: Matrix dimensions do not match for multiplication!" << endl;
        return {};
    }
    vector<vector<double>> result(rows_A, vector<double>(cols_B, 0.0));
    // 矩阵乘法计算
    for (int i = 0; i < rows_A; ++i) {
        for (int j = 0; j < cols_B; ++j) {
            for (int k = 0; k < cols_A; ++k) {
                result[i][j] += A[i][k] * B[k][j];
            }
        }
    }
    return result;
}
    \end{lstlisting}

\item 尝试调整上述c++代码的循环顺序:\\我们其实可以发现,
由于计算机一般是按行优先存储,所以在第三层
循环里,B[k][j]和B[k+1][j]在物理内存上距离较远,就很有可能造成
cache的miss,增加不必要的I/O时间。所以思路就是让k和j循环调换位置。

\begin{lstlisting}[language=c++]
//c++实现普通调整过顺序的矩阵乘法
// 调整后的循环顺序:i -> k -> j(其他的都和前面一样)
for (int i = 0; i < rows_A; ++i) {         
    for (int k = 0; k < cols_A; ++k) {      
        double a = A[i][k];                 // 缓存A[i][k],减少重复访问
        for (int j = 0; j < cols_B; ++j) {  // 遍历B的某一行的全部元素
            result[i][j] += a * B[k][j];    
        }
    }
}
    \end{lstlisting}
\item 循环展开\\
循环展开最明显的益处在于减少了循环次数,并且由于一次性取出很多变量,从而减少了内存访问。
展开的方式有多种,我首先尝试了对j做两路展开,k做四路展开:
\begin{lstlisting}[language=c++]
    //c++实现循环展开(j两路,k四路)
    ...(其他的都和前面展示的一样)
    for (int i = 0; i < rows_A; ++i) {
        int j = 0;
        // 每次处理2个j(列)以减少循环次数
        for (; j <= cols_B - 2; j += 2) {
            double sum1 = 0.0, sum2 = 0.0;
            int k = 0;
            // 每次处理4个k(展开因子=4)
            for (; k <= cols_A - 4; k += 4) {
                // 预加载A的元素
                const double a0 = A[i][k];
                const double a1 = A[i][k+1];
                const double a2 = A[i][k+2];
                const double a3 = A[i][k+3];

                // 为两个不同的j值计算乘积并累加
                sum1 += a0 * B[k][j] + a1 * B[k+1][j] + a2 * B[k+2][j] + a3 * B[k+3][j];
                sum2 += a0 * B[k][j+1] + a1 * B[k+1][j+1] + a2 * B[k+2][j+1] + a3 * B[k+3][j+1];
            }
            // 处理剩余k值
            for (; k < cols_A; ++k) {
                const double a = A[i][k];
                sum1 += a * B[k][j];
                sum2 += a * B[k][j+1];
            }
            result[i][j] = sum1;
            result[i][j+1] = sum2;
        }
        // 处理剩余j值
        for (; j < cols_B; ++j) {
            double sum = 0.0;
            int k = 0;
            // 同样展开k循环
            for (; k <= cols_A - 4; k += 4) {
                sum += A[i][k] * B[k][j] + A[i][k+1] * B[k+1][j] +
                       A[i][k+2] * B[k+2][j] + A[i][k+3] * B[k+3][j];
            }
            // 处理剩余k值
            for (; k < cols_A; ++k) {
                sum += A[i][k] * B[k][j];
            }
            result[i][j] = sum;
        }
    }
    ...
        \end{lstlisting}
我还尝试了对j做四路展开,k做八路展开:确实更快了,但是效果就没有那么立竿见影了
,由于原理是一样的,我就不把代码放到报告里了。
同时我还对调整过循环顺序的矩阵乘法做过循环展开,也不放在实验报告里了,对于调整过循环顺序的
矩阵乘法,只需要简单做一些展开就能达到不错的效果。

\item   Intel MKL:\\
这个代码已经给我们了,$cblas\_dgemm$函数可以执行
$C=alpha*A*B+beta*C$,
那我们只需要把alpha设置为1.0,beta设置为0就好了。
\begin{lstlisting}[language=c++]
//MKL
cblas_dgemm(CblasRowMajor, CblasNoTrans, CblasNoTrans,
    m, n, k, alpha, A, k, B, n, beta, C, n);
\end{lstlisting}

\end{itemize}


\section{实验结果}
\subsection{矩阵乘法的定义:}

\[ C_{ij} = \sum_{k=1}^{n} A_{ik} \times B_{kj} \]

\begin{itemize}
    \item 每个 \( C_{ij} \) 的计算需要:
    \begin{itemize}
        \item \( n \) 次乘法 \( (A_{ik} \times B_{kj}) \)
        \item \( n-1 \) 次加法(累加求和)
    \end{itemize}
    \item 因此,每个 \( C_{ij} \) 需要 \( 2n-1 \) 次浮点运算。
\end{itemize}

\subsection{整个矩阵 \( C \) 的计算:}
\begin{itemize}
    \item \( C \) 有 \( n \times n \) 个元素,因此总运算次数为:
    \[ n^2 \times (2n-1) = 2n^3 - n^2 \]
    \item 当 \( n \) 较大时(如 \( n \geq 100 \)),\( n^2 \) 相比 \( 2n^3 \) 可忽略不计,因此近似为:
    \[ \text{FLOPs} \approx 2n^3=2×1000^3=2 \times 10^9 \]
\end{itemize}
\subsection{理论峰值浮点性能 (Theoretical Peak FLOPS)}
公式如下:
\[ \text{Peak FLOPS} = \text{Num\_Cores} \times \text{Clock\_Speed (Hz)} \times \text{FLOPS\_per\_Cycle} \]

\begin{itemize}
    \item \textbf{Num\_Cores}:处理器核心数。
    \item \textbf{Clock\_Speed}:处理器主频(Hz)。
    \item \textbf{FLOPS\_per\_Cycle}:每个时钟周期能
    执行的浮点运算数。
    \begin{figure}[htbp]
        \centering
        \includegraphics[width=0.8\textwidth]{./figures/image2.png}
        \caption{CPU支持的指令集}
        \label{fig:AVX}
    \end{figure}
    \\AVX-512(Advanced Vector Extensions 512)是Intel推出的单指令多数据(SIMD)指令集扩展,旨在显著提升CPU的并行浮点和整数运算能力。它通过512位宽向量寄存器,允许单条指令同时处理多达16个单精度(32位)或8个双精度(64位)浮点数。
    由于支持AVX-512,每个周期能执行的浮点运算为512/64=8。\\
    \begin{figure}[htbp]
        \centering
        \includegraphics[width=0.8\textwidth]{./figures/image.png}
        \caption{笔记本电脑的CPU规格}
        \label{fig:CPU}
    \end{figure}
    我采用了最大睿频频率(见图2):
    \\故峰值性能为$4 × 4.2 × 10^9 \times 8=134,400,000,000$次每秒

\end{itemize}

\begin{table}[h]
    \centering
    \begin{tabular}{|c|c|c|p{1.5cm}|p{1.5cm}|c|p{2cm}|}
        \hline
        版本 & 实现描述 & 运行时间& 相对加速比 & 绝对加速比 & 浮点性能 & 峰值性能百分比 \\
        \hline
        1 & Python &199.45377s&—&1&10,027,386&0.00219\%\\ 
        \hline
        2 & C/C++ &19.387s&10.3&10.3&103,161,913&0.0768\% \\ 
        \hline
        3 & 调整循环顺序 &7.44672s&2.60&26.8&268,574,621&0.1998\% \\ 
        \hline
        4 & 循环展开(j四+k八)&4.58089s&1.63&43.5&436,596,382&0.3248\%\\ 
        \hline
        5 & 编译优化(Ofast) &1.86329s&2.46&107&1,073,370,221&0.799\%\\ 
        \hline
        6 & Intel MKL &0.017153s&109&11628&116,597,679,706&86.75\%\\ 
        \hline
        7 & Numpy &0.017150s&1&11629&116,618,075,801&86.77\%\\ 
        \hline
    \end{tabular}
\end{table}

\section{实验分析}
\subsection{浮点性能优化分析}
从上面的表格可以看出,如果不调用Intel MKL或numpy,我们峰值性能百分比是非常感人的。我认为原因有两个:
\begin{itemize}
    \item 首先我们的程序是跑在单核上的,这导致有三个核是在空闲的,效率拉满也只有25\%了。
    \item 其次是我们的程序没有使用AVX指令集,导致CPU中有大量的寄存器也是空闲的。
\end{itemize}
\subsection{编译优化分析}

    \begin{table}[h]
        \centering
        \begin{tabular}{|c|c|c|p{1.5cm}|p{1.5cm}|}
            \hline
            版本 & 实现描述 & 运行时间& 相对加速比 & 绝对加速比\\
            \hline
            1 & O &3.33031s&—&1\\ 
            \hline
            2 & O1 &3.56201s&0.93&0.93 \\ 
            \hline
            3 & O2 &1.89658s&1.88&1.76 \\ 
            \hline
            4 & O3&1.72805s&1.10&1.93\\ 
            \hline
            5 & Ofast&1.72292s&1&1.93\\
            \hline
            
        \end{tabular}
    \end{table}
根据上表,我们可以看到O和O1优化是差不多效果,而O2,O3,Ofast优化会快一些,但是彼此间差不多。
这说明O优化做的一些事情导致了速度的加快,而O1相比O优化做的事情则对程序的运行效率没有明显帮助。
O2,O3,Ofast同理。我猜测O和O1对于此程序的优化主要是循环展开,O2,O3,Ofast则是数组访问加速。
\end{document}