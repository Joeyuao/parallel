\documentclass{article}
\usepackage{xcolor}
\usepackage{listings}
\usepackage{hyperref}
\lstset{
    basicstyle=\ttfamily\footnotesize,
    keywordstyle=\color{blue},
    commentstyle=\color{green!60!black},
    stringstyle=\color{red},
    frame=single,
    breaklines=true,
    postbreak=\mbox{\textcolor{red}{$\hookrightarrow$}\space},
}

\begin{document}

\title{动态链接库(.so)的生成与调用指南}
\author{}
\date{}
\maketitle

\section{生成动态链接库}
\subsection{Makefile关键规则}
\begin{lstlisting}[language=make]
# 生成动态库
$(LIBRARY): parallel_for.o
    $(CC) -shared -o $@ $^
\end{lstlisting}

\begin{itemize}
    \item \texttt{-shared}:指定生成共享库
    \item \texttt{-o \$@}:输出文件名为目标名(如\texttt{libparallel\_for.so})
    \item \texttt{\$^}:自动引用所有依赖对象文件
\end{itemize}

\section{主程序链接动态库}
\subsection{Makefile链接规则}
\begin{lstlisting}[language=make]
$(TARGET): main.o
    $(CC) -o $@ $< -L. -lparallel_for -fopenmp -Wl,-rpath=.
\end{lstlisting}

\begin{tabular}{|l|l|}
\hline
\texttt{-L.} & 指定库搜索路径为当前目录 \\
\hline
\texttt{-lparallel\_for} & 链接\texttt{libparallel\_for.so}(省略前缀/后缀) \\
\hline
\texttt{-Wl,-rpath=.} & 设置运行时库搜索路径 \\
\hline
\texttt{-fopenmp} & 启用OpenMP支持 \\
\hline
\end{tabular}

\section{调用流程}
\subsection{C程序示例}
\begin{lstlisting}[language=C]
// main.c
extern void parallel_for(int start, int end, int step, void (*func)(int));

int main() {
    parallel_for(0, 100, 1, [](int i) {
        printf("%d\n", i);
    });
    return 0;
}
\end{lstlisting}

\subsection{编译运行步骤}
\begin{enumerate}
    \item 生成库和可执行文件:
    \begin{lstlisting}[language=bash]
    make
    \end{lstlisting}
    
    \item 检查依赖关系:
    \begin{lstlisting}[language=bash]
    ldd ./main
    \end{lstlisting}
    
    \item 运行程序:
    \begin{lstlisting}[language=bash]
    ./main
    \end{lstlisting}
\end{enumerate}

\section{注意事项}
\begin{itemize}
    \item 确保导出的函数未被\texttt{static}修饰
    \item 保持头文件声明与实现一致
    \item 推荐命名规范:\texttt{lib<name>.so}
    \item 开发调试建议使用相对路径(\texttt{-Wl,-rpath=.})
\end{itemize}

\end{document}