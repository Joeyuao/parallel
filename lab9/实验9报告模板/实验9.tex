%!TeX program = xelatex
\documentclass{SYSUReport}
\usepackage{tabularx} % 在导言区添加此行
\usepackage{float}
\usepackage{xcolor} % 已加载,无需重复
\usepackage{graphicx}
\usepackage{hyperref}
\lstset{
    basicstyle=\ttfamily\small,
    keywordstyle=\color{blue},
    commentstyle=\color{orange},
    stringstyle=\color{red},
    frame=single,
    breaklines=true,
    numbers=left,
    numberstyle=\tiny\color{gray}
}
% 根据个人情况修改
\headl{}
\headc{}
\headr{并行程序设计与算法实验}
\lessonTitle{并行程序设计与算法实验}
\reportTitle{Lab9-CUDA矩阵转置}
\stuname{李源卿}
\stuid{22336128}
\inst{计算机学院}
\major{计算机科学与技术}
\date{2025年5月21日}

\begin{document}

% =============================================
% Part 1: 封面
% =============================================
\cover
\thispagestyle{empty} % 首页不显示页码
\clearpage

% % =============================================
% % Part 4: 正文内容
% % =============================================
% % 重置页码,并使用阿拉伯数字
% \pagenumbering{arabic}
% \setcounter{page}{1}

%%可选择这里也放一个标题
%\begin{center}
%    \title{ \Huge \textbf{{标题}}}
%\end{center}

\section{实验目的}
\begin{itemize}
   \item 熟悉 CUDA 线程层次结构(grid、block、thread)和观察 warp 调度行为。 
    \item 掌握 CUDA 内存优化技术(共享内存、合并访问)。
    \item 理解线程块配置对性能的影响。
\end{itemize}
\section{实验内容}
\subsection{CUDA并行输出}
\begin{enumerate}
    \item 创建$n$个线程块,每个线程块的维度为$m\times k$。
    \item 每个线程均输出线程块编号、二维块内线程编号。例如:
    \begin{itemize}
        \item “Hello World from Thread (1, 2) in Block 10!”
         \item 主线程输出“Hello World from the host!”。 
        \item 在 main 函数结束前,调用 \texttt{cudaDeviceSynchronize()}。  
    \end{itemize}
    \item 完成上述内容,观察输出,并回答线程输出顺序是否有规律。 
\end{enumerate}

\subsection{使用 CUDA 实现矩阵转置及优化}
\begin{enumerate}
    \item 使用 CUDA 完成并行矩阵转置。 
    \item 随机生成 $N \times N$ 的矩阵 A。 
    \item 对其进行转置得到 $A^T$。 
    \item 分析不同线程块大小、矩阵规模、访存方式(全局内存访问,共享内存访问)、任务/数据划分和映射方式,对程序性能的影响。 
    \item 实现并对比以下两种矩阵转置方法:
    \begin{itemize}
        \item 仅使用全局内存的 CUDA 矩阵转置。 
        \item 使用共享内存的 CUDA 矩阵转置,并考虑优化存储体冲突。  
    \end{itemize}
\end{enumerate}
\section{实验结果与分析}
\subsection{CUDA Hello World 并行输出}
\subsubsection{实验现象}
描述实验观察到的现象,例如线程输出的顺序等。可以粘贴部分关键的运行截图或输出文本。\\
\begin{figure}
    \centering
    \includegraphics[width=0.5\textwidth]{figures/image0.png}
    \caption{HelloWorld 程序输出}
    \label{fig:HelloWorld}
\end{figure}
\textbf{关键代码}
\begin{lstlisting}[language=c]
__global__ void hello_world_kernel() {
    int x = threadIdx.x + blockIdx.x * blockDim.x;
    int y = threadIdx.y + blockIdx.y * blockDim.y;
    printf("Hello World from thread (%d, %d) in Block (%d)\n", threadIdx.x, threadIdx.y, blockIdx.x);
}
    ...
dim3 threadsPerBlock(16, 16);  
dim3 blocksInGrid(16);      
\end{lstlisting}
我们可以注重分析block编号为1的线程块:
\begin{itemize}
    \item 该block中线程输出按y坐标分组出现:先是y=10的所有x(0-15),然后是y=11的所有x(0-15)
    \item 每个y组内线程按x坐标顺序输出(从0到15)
    \item 与block 14的线程输出交错出现
\end{itemize}
\subsubsection{结果分析}
线程输出顺序是否有规律?为什么?结合CUDA线程调度机制进行解释。
\begin{itemize}
    \item \textbf{Warp组织线程}: 我们可以看到程序调用了y坐标为10和11的所有线程,总共32个,符合Warp的组织线程的数量。
    \\同时,我们发现程序调用了y坐标为10和11的所有线程(x编号为0-15),说明warp组织线程时是从x的方向将线程展平的。

    \item \textbf{非严格顺序调度}:虽然y=10和y=11的线程按顺序输出,但整体上block 1和block 14的线程输出交错出现,说明:
    \begin{itemize}
        \item GPU以warp为单位进行调度
        \item 执行顺序受硬件调度器控制,不保证严格的线性顺序
    \end{itemize}
    \item \textbf{SIMT执行特性}:同一warp内的线程执行相同指令(printf),但输出顺序可能受线程索引影响,体现了单指令多线程的执行特点。
\end{itemize}
\subsection{CUDA 矩阵转置及优化}
\subsubsection{正确性分析}
我选择输出前5行5列来验证正确性。\\

\begin{lstlisting}[language=c]
    ...

printMatrix(h_in, 5, n);

printMatrix(h_out, 5, n);//param:输出的矩阵、输出前几行几列、矩阵大小
    ...    
\end{lstlisting}
结果如下:
\begin{figure}
    \centering
    \includegraphics[width=0.5\textwidth]{figures/image1.png}
    \caption{trans 程序输出}
    \label{fig:trans}
\end{figure}
\subsubsection{不同实现方法的性能对比}
优化版本1是我自己想的,主要是更改访问顺序来解决存储体冲突。下面是我的思路:\\
存储体冲突的原因在于,一个存储体能存储32个矩阵元素,那么,如果按照图3中的箭头去依次分配元素给thread,就会造成一个warp中的16个线程访问同一个存储体,而另外16个线程访问
另外一个存储体,会产生冲突。\\
我们可以分析一下矩阵元素在存储体中的排布:(0,0)存储在bank0...(0,15)存储在bank15,(1,0)存储在bank16...(1,15)存储在bank31...(以此类推)。\\
刚才的问题在于,(0,0)和(2,0)在同一个存储体,分别被thread0和thread2同时访问。那么我们不妨改变一下访问的映射关系,把每一行的元素一个个分给thread们,也就是让thread0访问(0,0)(bank0),让thread1访问(1,0)(bank1)......我觉得这样就不会有存储体冲突了\\
所以有下面的代码:\\
\begin{lstlisting}[language=c]
int row = by + ty;
int col = bx + tx;
if (row < n && col < n) {
    smem[ty*BDIM + tx] = in[row * n + col];
}
__syncthreads();
if (row < n && col < n) {
    out[col * n + row] = smem[ty*BDIM + tx];
}    
\end{lstlisting}
优化版本2就是按照老师讲的思路:分配共享内存的时候多加一列,打乱矩阵元素在存储体中的排列。\\
\begin{figure}
    \centering
    \includegraphics[width=0.8\textwidth]{figures/image2.png}
    \caption{冲突原因}
    \label{fig:trans}
\end{figure}
下面我将展示不同矩阵转置实现(仅全局内存、使用共享内存、优化共享内存访问)在不同矩阵规模 (N) 和不同线程块大小下的运行时间。可以根据你的实验设置更改表格的矩阵规模、线程块大小。
\begin{table}[h!]
\centering
\caption{矩阵转置性能对比 (时间单位: ms)}
\begin{tabular}{|c|c|c|c|c|c|}
\hline
矩阵规模 (N) & 线程块大小 & 全局内存版本 & 共享内存版本 & 优化版本1 & 优化版本2 \\
\hline
\multirow{3}{*}{512} & 8$\times$8 &0.009873 &0.011587 &0.010315 &0.010779 \\
\cline{2-6} % 从第2列到第6列的线
& 16$\times$16 &0.011199 &0.010640 &0.010724 &0.010875 \\
\cline{2-6}
& 32$\times$32 &0.012310 &0.012183 &0.012963 &0.010320 \\
\hline
\multirow{3}{*}{1024} & 8$\times$8 &0.010590 &0.010228 &0.010749 & 0.010412\\
\cline{2-6}
& 16$\times$16 &0.010595 &0.011088 &0.010934 &0.010467 \\
\cline{2-6}
& 32$\times$32 &0.013219 &0.011646&0.012723 &0.010704 \\
\hline
\multirow{3}{*}{2048} & 8$\times$8 &0.009914 &0.010524 &0.010520 &0.010486 \\
\cline{2-6}
& 16$\times$16 &0.010818 &0.010747 &0.010572 &0.010557 \\
\cline{2-6}
& 32$\times$32 &0.012362 &0.012070 &0.012736 &0.010811 \\
\hline
\multirow{3}{*}{16384} & 8$\times$8 &0.013191&0.012427 &0.013074 &0.012300 \\
\cline{2-6}
& 16$\times$16 &0.014253 &0.011709 &0.013826 &0.011185 \\
\cline{2-6}
& 32$\times$32 &0.016478 &0.013745 &0.017241 &0.010062 \\
\hline
\end{tabular}
\end{table}

% \subsubsection{性能结果分析}
% 1. 根据实验结果,总结线程块大小、矩阵规模对程序性能的影响。哪种配置下性能最优?为什么?

% 回答:

% 2. 讨论任务/数据划分和映射方式对性能的影响。

% 回答:
\subsubsection{性能结果分析}
1. 根据实验结果,总结线程块大小、矩阵规模对程序性能的影响。哪种配置下性能最优?为什么?

\begin{itemize}
  \item \textbf{线程块大小对性能的影响}:
  \begin{itemize}
    \item 较小线程块(如 $8\times8$)在小规模矩阵(如 $N=512$ 或 $1024$)上表现更好。例如,在 $N=512$ 规模下,全局内存版本时间为 $0.009873$ ms。这是线程块小,分出来的线程块就多,线程块数目应当大于SM数量才能较好发挥GPU性能。
    \item 随着线程块增大(如 $32\times32$),在小规模矩阵上时间可能增加(如 $N=512$ 规模下全局内存版本时间达 $0.012310$ ms),表明块增大会导致块数目减少,有的SM甚至可能会空闲。
    \item 较大线程块(如 $32\times32$)在更大规模矩阵(如 $N=16384$)上优势显著,因为能更好地利用 GPU 的 warp 调度和并行资源,减少内核启动开销和块间同步成本。
  \end{itemize}
  
  \item \textbf{矩阵规模对性能的影响}:
  \begin{itemize}
    \item 随着矩阵规模增加(从 $N=512$ 到 $16384$),运行时间总体呈上升趋势。例如,全局内存版本在 $N=16384$ 规模下时间高达 $0.016478$ ms。
    \item 这是由于数据量增大导致全局内存访问同步延迟(由于存储体冲突需要同步线程)增加和并行任务复杂度提升。
    \item 优化版本(如优化版本 2)能有效缓解此影响,通过共享内存和优化访问模式、解决存储体冲突减轻延迟问题。
  \end{itemize}
  
  \item \textbf{最优配置}: 矩阵规模 $N=16384$、线程块大小 $32\times32$、优化版本 2(时间最低,$0.010062$ ms)。
  
  \item \textbf{原因}:
  \begin{itemize}
    \item 较大线程块与大矩阵匹配,线程块内的线程数和线程块的数目处于一个合适的大小,既可以充分被SM调用来隐藏延迟,SM内部的活跃线程数也较多。
    \item 优化版本 2 充分利用共享内存、减少 bank 冲突(通过调整数据访问模式确保合并访问),所以性能上较优。
  \end{itemize}
  \item \textbf{一个问题}:我还没有想明白我的方式为什么没有解决问题,得到的性能和没有解决存储体冲突类似甚至更差。
  \item \textbf{问题解决了}:我的方式解决了存储体冲突,但是顾此失彼,在写入时不是合并写入,导致更多的时间开销。
\end{itemize}

2. 讨论任务/数据划分和映射方式对性能的影响。

\begin{itemize}
  \item \textbf{任务/数据划分对性能的影响}:
  \begin{itemize}
    %\item 线程块大小的划分直接影响资源分配:较小块(如 $8\times8$) 导致更多线程块,增加内核启动开销和全局内存竞争(如 $N=512$ 规模下全局内存版本性能优于共享内存版本)。
    \item 较大块(如 $32\times32$) 能处理更大数据,减少块数量并提高计算密度(如在 $N=16384$ 规模下,$32\times32$ 线程块优化版本 2 性能最优)。
    \item 划分应与 GPU 架构特性(如共享内存大小和 warp 调度)精细匹配,才能最大化并行效率和内存带宽。
  \end{itemize}
  
  \item \textbf{映射方式对性能的影响}:
  \begin{itemize}
    \item 共享内存版本通常性能优于全局内存版本(如 $N=16384$ 规模、$16\times16$ 块下,共享内存版本时间 $0.011709$ ms vs. 全局内存版本 $0.014253$ ms),因为它减少了高延迟全局内存访问。
    \item 但不当的映射(如bank 冲突)会劣化性能(我的优化版本1可能是一个不恰当的映射)。
    \item 优化版本 2 通过改进数据映射(转置时使用共享内存填充以避免 bank 冲突)实现了较好的性能。
  \end{itemize}
\end{itemize}
% \subsection{矩阵转置性能分析}

% \subsubsection{线程块大小与矩阵规模的影响}
% 根据表1的实验数据,我们可以得出以下结论:

% \begin{itemize}
% \item \textbf{线程块大小影响}:对于小规模矩阵(512-2048),不同线程块大小对性能影响较小,差异在0.001-0.003ms范围内。但当矩阵规模增大到16384时,32×32线程块配置显示出明显优势,特别是在优化版本2中达到最佳性能(0.010062ms)。

% \item \textbf{矩阵规模影响}:随着矩阵规模增大,全局内存版本的性能下降最为明显(16384规模下比优化版本2慢约63.8\%),而优化版本2的性能下降幅度最小,显示出更好的可扩展性。

% \item \textbf{最优配置}:在16384矩阵规模下使用32×32线程块的优化版本2表现最佳。这主要归因于:
% \begin{enumerate}
% \item 更大的线程块能更好地利用SM(流式多处理器)中的计算资源
% \item 32×32线程块大小(1024线程)与GPU架构特性匹配良好,能充分利用warp调度机制
% \item 优化版本2采用了更高效的内存访问模式——合并内存访问,且解决了存储体冲突
% \end{enumerate}
% \end{itemize}

% \subsubsection{任务/数据划分和映射方式的影响}
% 任务和数据划分策略对性能有显著影响:

% \begin{itemize}
% \item \textbf{全局内存版本}:直接访问全局内存导致较高的延迟,特别是在大规模矩阵时性能下降明显,这验证了全局内存访问的高延迟特性。

% \item \textbf{共享内存版本}:通过使用共享内存作为缓存,减少了全局内存访问次数,在大多数情况下性能优于全局内存版本,特别是在16384规模下16×16线程块配置中比全局内存版本快约17.8%。

% \item \textbf{优化版本}:优化版本2表现最优,可能采用了以下优化策略:
% \begin{itemize}
% \item 更精细的数据划分,确保内存访问对齐
% \item 使用bank conflict-free的共享内存访问模式
% \item 合理的线程块维度设计,最大化内存访问的合并效果
% \item 可能使用了寄存器缓存等进一步优化手段
% \end{itemize}

% \item \textbf{映射方式}:实验结果表明,将矩阵元素映射到线程时,采用更大的线程块(如32×32)在大规模矩阵上更有效,这因为:
% \begin{itemize}
% \item 减少了线程块调度开销
% \item 提高了内存访问的局部性
% \item 更好地利用了SM内的并行计算资源
% \end{itemize}
% \end{itemize}
\end{document}