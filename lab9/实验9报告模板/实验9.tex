%!TeX program = xelatex
\documentclass{SYSUReport}
\usepackage{tabularx} % 在导言区添加此行
\usepackage{float}
% 根据个人情况修改
\headl{}
\headc{}
\headr{并行程序设计与算法实验}
\lessonTitle{并行程序设计与算法实验}
\reportTitle{Lab9-CUDA矩阵转置}
\stuname{xxx}
\stuid{xxx}
\inst{计算机学院}
\major{xxx}
\date{2025年5月21日}

\begin{document}

% =============================================
% Part 1: 封面
% =============================================
\cover
\thispagestyle{empty} % 首页不显示页码
\clearpage

% % =============================================
% % Part 4: 正文内容
% % =============================================
% % 重置页码,并使用阿拉伯数字
% \pagenumbering{arabic}
% \setcounter{page}{1}

%%可选择这里也放一个标题
%\begin{center}
%    \title{ \Huge \textbf{{标题}}}
%\end{center}

\section{实验目的}
\begin{itemize}
   \item 熟悉 CUDA 线程层次结构(grid、block、thread)和观察 warp 调度行为。 
    \item 掌握 CUDA 内存优化技术(共享内存、合并访问)。
    \item 理解线程块配置对性能的影响。
\end{itemize}
\section{实验内容}
\subsection{CUDA并行输出}
\begin{enumerate}
    \item 创建$n$个线程块,每个线程块的维度为$m\times k$。
    \item 每个线程均输出线程块编号、二维块内线程编号。例如:
    \begin{itemize}
        \item “Hello World from Thread (1, 2) in Block 10!”
         \item 主线程输出“Hello World from the host!”。 
        \item 在 main 函数结束前,调用 \texttt{cudaDeviceSynchronize()}。  
    \end{itemize}
    \item 完成上述内容,观察输出,并回答线程输出顺序是否有规律。 
\end{enumerate}

\subsection{使用 CUDA 实现矩阵转置及优化}
\begin{enumerate}
    \item 使用 CUDA 完成并行矩阵转置。 
    \item 随机生成 $N \times N$ 的矩阵 A。 
    \item 对其进行转置得到 $A^T$。 
    \item 分析不同线程块大小、矩阵规模、访存方式(全局内存访问,共享内存访问)、任务/数据划分和映射方式,对程序性能的影响。 
    \item 实现并对比以下两种矩阵转置方法:
    \begin{itemize}
        \item 仅使用全局内存的 CUDA 矩阵转置。 
        \item 使用共享内存的 CUDA 矩阵转置,并考虑优化存储体冲突。  
    \end{itemize}
\end{enumerate}
\section{实验结果与分析}
\subsection{CUDA Hello World 并行输出}
\subsubsection{实验现象}
描述实验观察到的现象,例如线程输出的顺序等。可以粘贴部分关键的运行截图或输出文本。
\begin{itemize}
    \item 回答:
\end{itemize}
\subsubsection{结果分析}
线程输出顺序是否有规律?为什么?结合CUDA线程调度机制进行解释。
\begin{itemize}
    \item 回答:
\end{itemize}
\subsection{CUDA 矩阵转置及优化}
\subsubsection{不同实现方法的性能对比}
1. 展示不同矩阵转置实现(仅全局内存、使用共享内存、优化共享内存访问)在不同矩阵规模 (N) 和不同线程块大小下的运行时间。可以根据你的实验设置更改表格的矩阵规模、线程块大小。
\begin{table}[h!]
\centering
\caption{矩阵转置性能对比 (时间单位: ms)}
\begin{tabular}{|c|c|c|c|c|}
\hline
矩阵规模 (N) & 线程块大小 & 全局内存版本 & 共享内存版本 & 优化共享内存版本 \\
\hline
\multirow{3}{*}{512} & 8$\times$8 & & & \\
\cline{2-5} % Draws a line from the 2nd to 5th column
& 16$\times$16 & & & \\
\cline{2-5}
& 32$\times$32 & & & \\
\hline
\multirow{3}{*}{1024} & 8$\times$8 & & & \\
\cline{2-5}
& 16$\times$16 & & & \\
\cline{2-5}
& 32$\times$32 & & & \\
\hline
\multirow{3}{*}{2048} & 8$\times$8 & & & \\
\cline{2-5}
& 16$\times$16 & & & \\
\cline{2-5}
& 32$\times$32 & & & \\
\hline
\end{tabular}
\end{table}
\subsubsection{结果分析}
1. 根据实验结果,总结线程块大小、矩阵规模对程序性能的影响。哪种配置下性能最优?为什么?

回答:

2. 讨论任务/数据划分和映射方式对性能的影响。

回答:

注:实验报告格式参考本模板,可在此基础上进行修改;实验代码以zip格式另提交;最终提交内容包括实验报告(pdf格式)和实验代码(zip压缩包格式)
\end{document}