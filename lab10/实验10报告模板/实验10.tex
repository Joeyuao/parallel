%!TeX program = xelatex
\documentclass{SYSUReport}
\usepackage{tabularx} % 在导言区添加此行
\usepackage{float}
\usepackage{graphicx}
% Listings configuration for CUDA C++
\lstdefinestyle{cuda_cpp}{
    language=C++,
    basicstyle=\ttfamily\footnotesize,
    keywordstyle=\color{blue},
    commentstyle=\color{green!40!black},
    stringstyle=\color{purple},
    numbers=left,
    numberstyle=\tiny\color{gray},
    breaklines=true,
    breakatwhitespace=true,
    tabsize=2,
    captionpos=b, % Caption at the bottom
    escapeinside={\%*}{*)}, % Allows LaTeX commands inside listings
    morekeywords={__global__, __shared__, threadIdx, blockIdx, blockDim, gridDim, cudaMalloc, cudaMemcpy, cudaFree, __syncthreads, dim3} % Add CUDA specific keywords
}
% 根据个人情况修改
\headl{}
\headc{}
\headr{并行程序设计与算法实验}
\lessonTitle{并行程序设计与算法实验}
\reportTitle{Lab10-CUDA并行矩阵乘法}
\stuname{李源卿}
\stuid{22336128}
\inst{计算机学院}
\major{计算机科学与技术}
\date{2025年5月28日}

\begin{document}

% =============================================
% Part 1: 封面
% =============================================
\cover
\thispagestyle{empty} % 首页不显示页码
\clearpage

% % =============================================
% % Part 4: 正文内容
% % =============================================
% % 重置页码,并使用阿拉伯数字
% \pagenumbering{arabic}
% \setcounter{page}{1}

%%可选择这里也放一个标题
%\begin{center}
%    \title{ \Huge \textbf{{标题}}}
%\end{center}

\section{实验目的}
\begin{itemize}
    \item 理解CUDA编程模型(Grid、Block、Thread)及其在矩阵乘法中的应用。
    \item 学习GPU内存优化技术。
\end{itemize}
\section{实验内容}
\begin{itemize}
    \item 实现基础矩阵乘法
\item 优化矩阵乘法:共享内存,分块技术
\item 测量不同实现的运行时间
\end{itemize}
\section{实验结果与分析}
\subsection{不同实现方法的性能对比}
\begin{table}[H] % Changed from [h!] to [H] for 'here definitely' via float package
\centering
\caption{按行划分:性能对比 (时间单位: ms)}
\label{tab:perf_combined}
\begin{tabular}{|c|c|c|c|c|}
\hline
矩阵规模 (N) & 线程块大小 & 朴素实现 & 基于共享内存优化 & 基于寄存器分块优化 \\
\hline
\multirow{3}{*}{512} & 8$\times$8 &       &       &       \\ % Students fill these
\cline{2-5}
& 16$\times$16 &       &       &       \\
\cline{2-5}
& 32$\times$32 &       &       &       \\
\hline
\multirow{3}{*}{1024} & 8$\times$8 &       &       &       \\
\cline{2-5}
& 16$\times$16 &       &       &       \\
\cline{2-5}
& 32$\times$32 &       &       &       \\
\hline
\multirow{3}{*}{2048} & 8$\times$8 &       &       &       \\
\cline{2-5}
& 16$\times$16 &       &       &       \\
\cline{2-5}
& 32$\times$32 &       &       &       \\
\hline
% 可以根据需要添加更多矩阵规模或线程块大小的测试
\end{tabular}
\end{table}
\begin{table}[H] % Changed from [h!] to [H] for 'here definitely' via float package
\centering
\caption{按列划分:性能对比 (时间单位: ms)}
\label{tab:perf_combined}
\begin{tabular}{|c|c|c|c|c|}
\hline
矩阵规模 (N) & 线程块大小 & 朴素实现 & 基于共享内存优化 & 基于寄存器分块优化 \\
\hline
\multirow{3}{*}{512} & 8$\times$8 &       &       &       \\ % Students fill these
\cline{2-5}
& 16$\times$16 &       &       &       \\
\cline{2-5}
& 32$\times$32 &       &       &       \\
\hline
\multirow{3}{*}{1024} & 8$\times$8 &       &       &       \\
\cline{2-5}
& 16$\times$16 &       &       &       \\
\cline{2-5}
& 32$\times$32 &       &       &       \\
\hline
\multirow{3}{*}{2048} & 8$\times$8 &       &       &       \\
\cline{2-5}
& 16$\times$16 &       &       &       \\
\cline{2-5}
& 32$\times$32 &       &       &       \\
\hline
% 可以根据需要添加更多矩阵规模或线程块大小的测试
\end{tabular}
\end{table}
\begin{table}[H] % Changed from [h!] to [H] for 'here definitely' via float package
\centering
\caption{按数据块划分:性能对比 (时间单位: ms)}
\label{tab:perf_combined}
\begin{tabular}{|c|c|c|c|c|}
\hline
矩阵规模 (N) & 线程块大小 & 朴素实现 & 基于共享内存优化 & 基于寄存器分块优化 \\
\hline
\multirow{3}{*}{512} & 8$\times$8 &       &       &       \\ % Students fill these
\cline{2-5}
& 16$\times$16 &       &       &       \\
\cline{2-5}
& 32$\times$32 &       &       &       \\
\hline
\multirow{3}{*}{1024} & 8$\times$8 &       &       &       \\
\cline{2-5}
& 16$\times$16 &       &       &       \\
\cline{2-5}
& 32$\times$32 &       &       &       \\
\hline
\multirow{3}{*}{2048} & 8$\times$8 &       &       &       \\
\cline{2-5}
& 16$\times$16 &       &       &       \\
\cline{2-5}
& 32$\times$32 &       &       &       \\
\hline
% 可以根据需要添加更多矩阵规模或线程块大小的测试
\end{tabular}
\end{table}
%其他划分方式可自由设计表格
分析性能差异的原因:
\begin{itemize}
    \item 结合CUDA内存模型和矩阵乘法原理,分析造成观察到的性能差异的可能原因。\par
    回答:
    \item 如何选择合适的线程块大小以提高占用率?\par
    回答
    \item 思考如果按不同的方式划分(例如,按行、列、数据块划分),可能会对性能和实现复杂度带来什么影响?\par
    回答:
    \item 何时应该优先考虑使用哪种存储?\par
    回答:


\end{itemize}

\end{document}