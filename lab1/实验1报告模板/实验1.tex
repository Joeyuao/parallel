%!TeX program = xelatex
\documentclass{SYSUReport}
\usepackage{tabularx} % 在导言区添加此行
\usepackage{graphicx}
\lstset{
    basicstyle=\ttfamily\small,
    keywordstyle=\color{blue},
    commentstyle=\color{orange},
    stringstyle=\color{red},
    frame=single,
    breaklines=true,
    numbers=left,
    numberstyle=\tiny\color{gray}
}
\usepackage{array}
\usepackage{booktabs}
\usepackage{caption}
% 根据个人情况修改
\headl{}
\headc{}
\headr{并行程序设计与算法实验}
\lessonTitle{并行程序设计与算法实验}
\reportTitle{Lab1-基于MPI的并行矩阵乘法}
\stuname{李源卿}
\stuid{22336128}
\inst{计算机学院}
\major{计算机科学与技术}
\date{2025年3月26日}

\begin{document}

% =============================================
% Part 1: 封面
% =============================================
\cover
\thispagestyle{empty} % 首页不显示页码
\clearpage

% % =============================================
% % Part 4: 正文内容
% % =============================================
% % 重置页码,并使用阿拉伯数字
% \pagenumbering{arabic}
% \setcounter{page}{1}

%%可选择这里也放一个标题
%\begin{center}
%    \title{ \Huge \textbf{{标题}}}
%\end{center}

\section{实验目的}
\begin{itemize}
   \item 掌握 MPI 程序的编译和运行方法。
    \item 理解 MPI 点对点通信的基本原理。
    \item 了解 MPI 程序的 GDB 调试流程。
\end{itemize}

\section{实验内容}
\begin{itemize}
   \item 使用 MPI 点对点通信实现并行矩阵乘法。
    \item 设置进程数量(1$\sim$16)及矩阵规模(128$\sim$2048)。
    \item 根据运行时间,分析程序的并行性能。
\end{itemize}

\section{实验步骤}
\subsection{使用MPI进行矩阵乘法}
这两个函数在本次实验中可以承担所有的通讯职能,并且使用方法简单。
需要达到类似广播的功能只需要在根进程
用for循环对每个进程都使用Send发送一个消息,然后在其他进程调
用Recv接收就好了。
收集结果的时候也使用for循环达到了类似Scatter的效果。
\begin{lstlisting}[language=c++]
#include <mpi.h>
#include <stdio.h>
#include <stdlib.h>
#include <string.h>
// A:m*k B:k*n C:m*n
void matrix_multiply(double *A, double *B, double *C, int rows, int n, int kk) {
    for (int i = 0; i < rows; ++i) {          
        for (int k = 0; k < kk; ++k) {     
            double a = A[i*kk+k];                
            for (int j = 0; j < n; ++j) {  
                C[i*n+j] += a * B[k*n+j];   
            }
        }
    }
}
void pirnt_mat(double* A,int m,int n){
    for (int i = 0; i < m; i++)
    {
        for (int j = 0; j < n; j++)
        {
            printf("%lf ",A[i*n+j]);
        }
        printf("\n");
    }
    
}

int main(int argc,char **argv){
    // printf("%d\n",argc);
    // printf("%s\n",argv[2]);
    //初始化
    MPI_Init(&argc, &argv);
    double *C = NULL;
    int rank, p;
    MPI_Comm_rank(MPI_COMM_WORLD, &rank);
    MPI_Comm_size(MPI_COMM_WORLD, &p);
    int m , n , k ; 
    double *A = NULL;
    double *B = NULL;
    
    if(rank==0){
        //输入维度
        printf("input the m,n,k:\n");
        scanf("%d%d%d",&m,&n,&k);
        int mnk[3]={m,n,k};
        MPI_Bcast(mnk,3,MPI_INT,0,MPI_COMM_WORLD);
        m=mnk[0];n=mnk[1];k=mnk[2];
        //初始化A和B矩阵
        A=(double*)malloc(m*k*sizeof(double));
        B=(double*)malloc(k*n*sizeof(double));
        C=(double*)malloc(m*n*sizeof(double));
        for (int i = 0; i < m*k; i++)
        {
            A[i]=i+1;
        }
        for (int i = 0; i < k*n; i++)
        {
            B[i]=i+1;
        }
        //规划数据的分组
        int *rows_per_rank = (int *)malloc(p * sizeof(int));
        int *start_row_per_rank = (int *)malloc(p * sizeof(int));
        int remainder = m % p;
        int base_rows = m / p;
        int current_start = 0;
        for (int i = 0; i < p; i++)
        {
            //前remainder组多算一行,刚好不多余
            rows_per_rank[i]=(i<remainder)?base_rows+1:base_rows;
            start_row_per_rank[i]=current_start;
            current_start+=rows_per_rank[i];
        }
        //发送切好片的A矩阵和整个B矩阵,发rows_per_rank[i]是因为接收方需要用来接收矩阵A和发送矩阵C
        for (int i = 1; i < p; i++)
        {
            MPI_Send(&rows_per_rank[i],1,MPI_INT,i,0,MPI_COMM_WORLD);
            MPI_Send(A+start_row_per_rank[i]*k,rows_per_rank[i]*k,MPI_DOUBLE,i,1,MPI_COMM_WORLD);
            MPI_Send(B,k*n,MPI_DOUBLE,i,2,MPI_COMM_WORLD);
        }
        //矩阵乘(调整过循环顺序)
        matrix_multiply(A,B,C,rows_per_rank[0],n,k);
        for (int i = 1; i < p; i++)
        {
            //收集结果
            MPI_Recv(C+start_row_per_rank[i]*n,rows_per_rank[i]*n,MPI_DOUBLE,i,0,MPI_COMM_WORLD,MPI_STATUS_IGNORE);
        }
        pirnt_mat(C,m,n);
        free(A);
        free(B);
        free(C);
        free(rows_per_rank);
        free(start_row_per_rank);
    }
    else{
        //获取矩阵维度
        int mnk[3]={};
        MPI_Bcast(mnk,3,MPI_INT,0,MPI_COMM_WORLD);
        m=mnk[0];n=mnk[1];k=mnk[2];
        //获取当前进程需要算的行数
        int local_rows=0;
        MPI_Recv(&local_rows,1,MPI_INT,0,0,MPI_COMM_WORLD,MPI_STATUS_IGNORE);
        A=(double*)malloc(local_rows*k*sizeof(double));
        B=(double*)malloc(n*k*sizeof(double));
        double* C=(double*)malloc(local_rows*n*sizeof(double));
        //接受矩阵A,B
        MPI_Recv(A,local_rows*k,MPI_DOUBLE,0,1,MPI_COMM_WORLD,MPI_STATUS_IGNORE);
        MPI_Recv(B,n*k,MPI_DOUBLE,0,2,MPI_COMM_WORLD,MPI_STATUS_IGNORE);
        matrix_multiply(A,B,C,local_rows,n,k);
        //返回结果给根进程
        MPI_Send(C,local_rows*n,MPI_DOUBLE,0,0,MPI_COMM_WORLD);
        free(A);
        free(B);
        free(C);
    }
    
    MPI_Finalize();
    return 0;
}
\end{lstlisting}

% \subsubsection{仅使用\texttt{MPI_Send}和\texttt{MPI_Recv}}


\section{实验结果}
\subsection{计时}
\begin{itemize}
    \item 由于我暂时没有调整进程的上限个数,所以运行环境是在超算习堂
    \item 测试时间的函数使用了MPI\_Wtime,这是一个高分辨率、经过 (或墙) 时钟。
    单位为s。
    \item 测试结果是统计的最慢的进程所用的时间
\end{itemize}
计时函数参考:https://learn.microsoft.com/zh-cn/message-passing-interface/mpi-wtime-function\\
测试代码:
\begin{lstlisting}[language=c++]
    if(rank==0){
        //输入维度
        printf("input the m,n,k:\n");
        scanf("%d%d%d",&m,&n,&k);
    }
    MPI_Barrier(MPI_COMM_WORLD);
    double start=MPI_Wtime();
    ......
    double time=MPI_Wtime()-start;
    printf("process %d costs %lfs\n",rank,time);
    double max_time;
    MPI_Reduce(&time,&max_time,1,MPI_DOUBLE,MPI_MAX,0,MPI_COMM_WORLD);
    if(rank==0){
        printf("the max time: %lfs\n",max_time);
    }
    MPI_Finalize();
    return 0;
\end{lstlisting}
\subsection{运行时间}
% Please add the following required packages to your document preamble:
% \usepackage{booktabs}
% \usepackage{multirow}
% \usepackage{graphicx}
% Please add the following required packages to your document preamble:
% \usepackage{booktabs}
% \usepackage{multirow}
% \usepackage{graphicx}
% Please add the following required packages to your document preamble:
% \usepackage{multirow}
% \usepackage{graphicx}
% Please add the following required packages to your document preamble:
% \usepackage{multirow}
\begin{table}[htpb]
\centering
\begin{tabular}{|c|lllll|}
\hline
\multirow{2}{*}{进程数} & \multicolumn{5}{c|}{矩阵规模}                                                                        \\ \cline{2-6} 
 & \multicolumn{1}{l|}{128} & \multicolumn{1}{l|}{256} & \multicolumn{1}{l|}{512} & \multicolumn{1}{l|}{1024} & 2048 \\ \hline
1                    & \multicolumn{1}{l|}{0.012775s} & \multicolumn{1}{l|}{0.082896s} & \multicolumn{1}{l|}{0.643198s} & \multicolumn{1}{l|}{5.206706s} &41.496786s\\ \hline
2                    & \multicolumn{1}{l|}{0.005949s} & \multicolumn{1}{l|}{0.042818s} & \multicolumn{1}{l|}{0.324621s} & \multicolumn{1}{l|}{2.629395s} &21.415073s\\ \hline
4                    & \multicolumn{1}{l|}{0.004561} & \multicolumn{1}{l|}{0.024158s} & \multicolumn{1}{l|}{0.172887s} & \multicolumn{1}{l|}{1.381328s} &10.611006s\\ \hline
8                   & \multicolumn{1}{l|}{0.004862s} & \multicolumn{1}{l|}{0.015255s} & \multicolumn{1}{l|}{0.255603s} & \multicolumn{1}{l|}{1.826807s} &11.766949s\\ \hline
16                   & \multicolumn{1}{l|}{0.106172s} & \multicolumn{1}{l|}{0.017869s} & \multicolumn{1}{l|}{0.496006s} & \multicolumn{1}{l|}{2.704538s} &18.485188s\\ \hline
\end{tabular}
\end{table}
\subsection{结果分析}
我们可以看到,随着进程数的增多,运行时间也不一定越快,结合课上所学和自己的思考,我觉得主要是以下两点原因:
\begin{itemize}
   \item 超算习堂的实验环境中,真实的物理核数是4,
   那么当进程数超过物理核之后,系统就会对多个进程进行调度,让这些进程
   轮流上处理器运行。所以当进程数小于等于物理核对时候,加速比和我们所设想的增长
   几乎一致,两个核快两倍,四个核快四倍,但是多余四个核表现不佳,这是因为处于运行中的进程有限,其他的大多处于挂起状态。
   \item 其次我考虑到的原因是进程间的通信开销,因为进程越多,需要通信的次数也越多,通信的开销也会变大。 
\end{itemize}
我们横向对比前三行(进程数为1,2,4),会发现每一行的前一个耗时是后一个耗时的8倍左右,这是符合我们预期的。
因为矩阵乘法是$O(n^3)$开销,那么$n->2n$,总开销就扩大8倍。
\\当进程数多于物理核的时候,时间开销就和操作系统对进程的调度有更强的关系了,所以8倍关系不是特别明显。
\section{讨论题}
\subsection{在内存受限情况下,如何进行大规模矩阵乘法计算?}
    回答:在内存受限的情况下,如果我们要进行大规模的矩阵乘法,
    那么我们需要特别关注的地方一定是磁盘I/O时间,因为内存受限
    ,我们的内存可能无法存下太多数据,有部分矩阵的数据就要从磁盘中读取,
    这往往是很费时间的操作;其次,我们也可以关注cache的命中率(可以使用循环展开,调整循环顺序等)。
    \begin{itemize}
        \item 矩阵分块:将大矩阵A(m×n)和B(n×p)划分为适合内存的子块。例如:
            \subitem  将A按行划分为多个大小为 $m_b \times n$ 的块。
            \subitem  将B按列划分为多个大小为 $n \times p_b$ 的块。
            \subitem  将结果C存储为$m_b \times p_b$大小的块。
            \subitem  假设矩阵中元素是double,那么要保证
            \\ \quad \quad \quad \quad \quad$(m_b*n+n*p_b+m_b*p_b)*sizeof(double)\leq$内存大小
        \item 可以固定已经加载进入内存的A或者B的块,若固定了B的块(即B的某几列),
        那么我们可以依次加载A的所有块进入内存,即只替换A的块,以此来减少I/O次数。
        \item 按行或列顺序存储矩阵数据,减少磁盘寻道时间。例如,将A的行块连续存储,B的列块按访问顺序存储(列优先存储),这样可以让读入内存的块有尽可能多的我们想要的数据。
        \item 结合循环展开,循环顺序调整,以及SIMD指令来计算C矩阵中元素来加速计算。
        \item 结合Strassen等快速矩阵乘法算法,减少计算量,间接降低内存压力。
    \end{itemize}
\subsection{如何提高大规模稀疏矩阵乘法性能?}

    回答:\\
    用CSR格式压缩存储矩阵A和矩阵B:
    \begin{itemize}
        \item row\_ptr:长度为m+1(m为行数),row\_ptr[i]到row\_ptr[i+1]-1为第i行的非零元素索引。
        \item col\_indices:非零元素的列索引。
        \item values:非零元素的值。
    \end{itemize}
    此时我们应该把矩阵B也按照行优先存储,因为下面的算法是调整过循环顺序的(也就是我们不会在第三次循环中直接算出$C_{i,j}$)

    \begin{lstlisting}[language=c++]
        // 假设C初始化为全零的密集矩阵
    for (int i = 0; i < A_rows; i++) {
        for (int k_ptr = A.row_ptr[i]; k_ptr < A.row_ptr[i+1]; k_ptr++) {
            int k = A.col_indices[k_ptr];
            float A_ik = A.values[k_ptr];
            for (int j_ptr = B.row_ptr[k]; j_ptr < B.row_ptr[k+1]; j_ptr++) {
                int j = B.col_indices[j_ptr];
                float B_kj = B.values[j_ptr];
                C[i][j] += A_ik * B_kj; // 累加到结果矩阵
            }
        }
    }
    
    \end{lstlisting}
    这样我们就在牺牲了一部分空间的代价下,跳过了对矩阵元素0的运算。
\end{document}