\documentclass{SYSUReport}
\usepackage{tabularx} % 在导言区添加此行
\usepackage{graphicx}
\lstset{
    basicstyle=\ttfamily\small,
    keywordstyle=\color{blue},
    commentstyle=\color{orange},
    stringstyle=\color{red},
    frame=single,
    breaklines=true,
    numbers=left,
    numberstyle=\tiny\color{gray}
}
\usepackage{array}
\usepackage{booktabs}
\usepackage{caption}
% 根据个人情况修改
\headl{}
\headc{}
\headr{并行程序设计与算法实验}
\lessonTitle{并行程序设计与算法实验}
\reportTitle{Lab2-基于MPI的并行矩阵乘法(进阶)}
\stuname{李源卿}
\stuid{22336128}
\inst{计算机学院}
\major{计算机科学与技术}
\date{2025年4月2日}

\begin{document}

% =============================================
% Part 1: 封面
% =============================================
\cover
\thispagestyle{empty} % 首页不显示页码
\clearpage

% =============================================
% Part 4: 正文内容
% =============================================
% 重置页码,并使用阿拉伯数字
\pagenumbering{arabic}
\setcounter{page}{1}

% 可选择这里也放一个标题
%\begin{center}
%    \title{ \Huge \textbf{{标题}}}
%\end{center}

\section{实验目的}
\begin{itemize}
    \item 掌握MPI集合通信在并行矩阵乘法中的应用
    \item 学习使用MPI\_Type\_create\_struct创建派生数据类型
    \item 分析不同通信方式和任务划分对并行性能的影响
    \item 研究并行程序的扩展性和性能优化方法
\end{itemize}

\section{实验内容}
\begin{itemize}
    \item 使用MPI集合通信实现并行矩阵乘法
    \item 使用MPI\_Type\_create\_struct聚合进程内变量后通信
    \item 选做:尝试不同数据/任务划分方式
    \begin{itemize}
        \item 请描述你的数据/任务划分方式。
        \item 回答:......
    \end{itemize}
\end{itemize}

\section{代码说明}
\subsection{SPMD}
与之前使用Recv和Send不同的是,Bcast,Scatterv和Gatherv需要每个进程都调用。
所以没有使用如下格式:
\begin{lstlisting}[language=c++]
    if(rank==0){
        ...
    }
    else{
        ...
    }
\end{lstlisting}
而是采取:
\begin{lstlisting}[language=c++]
    MPI_Bcast(...);
    if(rank==0){
        ...
    }
    MPI_Scatterv(...);
\end{lstlisting}
\subsection{Scatterv和Gatherv}
Scatterv和Gatherv的好处在于可以将不同长度的数据散射到各个进程,
在无法均匀分配实验数据时,我们可以用这两个函数来实现集合通讯。
\begin{lstlisting}[language=c++]
rows_per_process = (int *)malloc(size * sizeof(int));
displs = (int *)malloc(size * sizeof(int));
recvcounts = (int *)malloc(size * sizeof(int));
// 规划数据分组
int base_rows = m / size;
int remainder = m % size;
int cur_row = 0;
if (rank == 0) {
    for (int i = 0; i < size; i++) {
        // 前 remainder 个进程多处理一行
        rows_per_process[i] = base_rows + (i < remainder ? 1 : 0);
        displs[i] = cur_row * k; //偏移量
        cur_row += rows_per_process[i];
        recvcounts[i] = rows_per_process[i] * k; // 每个进程接收的元素数
    }
}
...
MPI_Scatterv(A, recvcounts, displs, MPI_DOUBLE,
        A_local, recvcounts[rank], MPI_DOUBLE,
        0, MPI_COMM_WORLD);
//A-发送缓冲区 recvcounts[i]表示第i个进程接收的数组大小displs[i]表示每个进程接收的数据在A中的起始位置。
//0表示0是根进程,像MPI_DOUBLE则是数据类型,MPI_COMM_WORLD是通讯集合
...
MPI_Gatherv(C_local, rows_per_process[rank] * n, MPI_DOUBLE,
        C, recvcounts, displs, MPI_DOUBLE,
        0, MPI_COMM_WORLD);
//C_local 待聚合的本地缓冲区(发送缓冲区),rows_per_process[rank] * n为发送数据量,c为接收缓冲区,recvcounts是一个存储了各个进程发送量的数组,displs决定了每个进程发送来的数据将在C的什么位置开始记录。
\end{lstlisting}
\subsection{MPI\_Type\_create\_struct}
MPI\_Type\_create\_struct的优势在于可以把不同类型的数据打包,但是在本次实验中不涉及打包
不同类型的数据,于是我自定义了三个类型的数据:
\subsubsection{mkn\_type}
\begin{lstlisting}[language=c++]
typedef struct {
    int m;
    int k;
    int n;
} mkn;
MPI_Datatype mkn_type;
{
    int blocklengths[3] = {1, 1, 1};
    MPI_Datatype types[3] = {MPI_INT, MPI_INT, MPI_INT};
    MPI_Aint displacements[3];
    displacements[0] = 0;
    displacements[1] = 4;
    displacements[2] = 8;
    MPI_Type_create_struct(3, blocklengths, displacements, types, &mkn_type);
    MPI_Type_commit(&mkn_type);
}
\end{lstlisting}

\subsubsection{size\_len\_block}
以上是比较标准的流程,对于不同类型的数据很有必要,因为每个数据的偏移量可能不同。
但是对于相同的数据,像下面这样创建更加方便:
\begin{lstlisting}[language=c++]
MPI_Datatype size_len_block;
{
    int blocklengths = size;
    MPI_Datatype types = MPI_INT;
    MPI_Aint displacements = 0;
    //这里都需要取地址
    MPI_Type_create_struct(1, &blocklengths, &displacements, &types, &size_len_block);
    MPI_Type_commit(&size_len_block);
}
\end{lstlisting}

我还打包了矩阵B,不过和上面的size\_len\_block一致,就不放到报告里了。

\section{实验结果}
\subsection{正确性}
之前的实验其实我都验证过,但是没有在报告里指出。
此处用9×9的矩阵做验证。
\begin{figure}[htbp]
    \centering
    \includegraphics[width=1.0\textwidth]{figures/image.png}
    \caption{numpy结果}
    \label{fig:example}
\end{figure}
\begin{figure}[htbp]
    \centering
    \includegraphics[width=1.0\textwidth]{figures/image1.png}
    \caption{程序结果}
    \label{fig:example1}
\end{figure}
\subsection{性能分析}
根据运行结果,填入下表以记录不同线程数量和矩阵规模下的运行时间:
\begin{table}[htbp]
    \centering
    \caption{用MPI集合通信实现}
    \label{表1}
    \begin{tabular}{|c|l|l|l|l|l|}
        \hline
        \multirow{2}{*}{进程数} & \multicolumn{5}{c|}{矩阵规模} \\ \cline{2-6} 
        & 128 & 256 & 512 & 1024 & 2048 \\ \hline
        1 &0.012799s&0.098125s& 0.778544s&6.149306s&44.429201s\\ \hline
        2 &0.007039s&0.051697s&0.392966s&2.658302s&20.433961s\\ \hline
        4 &0.004355s&0.027613s&0.220402s&1.378771s&10.776583s\\ \hline
        8 &0.005023s&0.016130s&0.372274s&1.538535s&10.776583s\\ \hline
        16 &0.004399s&0.104787s&0.493304s&2.714182s&20.777025s\\ \hline
    \end{tabular}
\end{table}

\begin{table}[htbp]
    \centering
    \caption{用MPI\_Type\_create\_struct聚合进程内变量}
    \label{表2}
    \begin{tabular}{|c|l|l|l|l|l|}
        \hline
        \multirow{2}{*}{进程数} & \multicolumn{5}{c|}{矩阵规模} \\ \cline{2-6} 
        & 128 & 256 & 512 & 1024 & 2048 \\ \hline
        1 &0.012969s&0.099214s&0.776359s&6.149796s&40.722858s\\ \hline
        2 &0.007144s&0.051306s&0.397161s&2.688867s&20.630480s\\ \hline
        4 &0.004386s&0.023349s&0.202343s&1.347274s&10.715022s\\ \hline
        8 &0.005222s&0.016376s&0.197371s&1.472587s&11.182086s\\ \hline
        16 &0.005011s&0.092633s&0.397388s&2.615108s&19.298225s\\ \hline
    \end{tabular}
\end{table}

\begin{table}[htbp]
    \centering
    \caption{选做题实验结果请填写在此表}
    \label{表3}
    \begin{tabular}{|c|l|l|l|l|l|}
        \hline
        \multirow{2}{*}{进程数} & \multicolumn{5}{c|}{矩阵规模} \\ \cline{2-6} 
        & 128 & 256 & 512 & 1024 & 2048 \\ \hline
        1 &  &  &  &  &  \\ \hline
        2 &  &  &  &  &  \\ \hline
        4 &  &  &  &  &  \\ \hline
        8 &  &  &  &  &  \\ \hline
        16 &  &  &  &  &  \\ \hline
    \end{tabular}
\end{table}

\section{实验分析}
根据运行时间,分析程序并行性能及扩展性
\begin{itemize}
    \item 使用MPI集合通信实现并行矩阵乘法:是强可拓展的。问题规模为128情况下,当进程数扩大为原来两倍的时候,加速比都约为1.7,
    效率几乎没变,其余情况加速比都约为2。所以是强可拓展的。
    \item 使用MPI\_Type\_create\_struct聚合进程内变量后通信:是强可拓展的,当核数加倍时,
    时间缩短为原来的1/2左右(规模为128时,问题规模增加为原来2倍,加速比约为1.7,效率保持稳定)。另外在性能方面我们可以看进程数为16时,使用普通集合通信和使用
    MPI\_Type\_create\_struct的对比,我们会发现使用MPI\_Type\_create\_struct的时间会稍微短
    一些,这说明虽然创建自定义类型需要耗费一些时间,但是也能带来一些收益。
    \item 你的方法(选做):
\end{itemize}
\end{document}