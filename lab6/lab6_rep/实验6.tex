%!TeX program = xelatex
\documentclass{SYSUReport}
\usepackage{tabularx} % 在导言区添加此行
\usepackage{float}
\usepackage{xcolor} % 已加载,无需重复
\usepackage{graphicx}
\usepackage{hyperref}
\lstset{
    basicstyle=\ttfamily\small,
    keywordstyle=\color{blue},
    commentstyle=\color{orange},
    stringstyle=\color{red},
    frame=single,
    breaklines=true,
    numbers=left,
    numberstyle=\tiny\color{gray}
}
% 根据个人情况修改
\headl{}
\headc{}
\headr{并行程序设计与算法实验}
\lessonTitle{并行程序设计与算法实验}
\reportTitle{Lab6-Pthreads并行构造}
\stuname{李源卿}
\stuid{22336128}
\inst{计算机学院}
\major{计算机科学与技术}
\date{2025年5月10日}

\begin{document}

% =============================================
% Part 1: 封面
% =============================================
\cover
\thispagestyle{empty} % 首页不显示页码
\clearpage

% % =============================================
% % Part 4: 正文内容
% % =============================================
% % 重置页码,并使用阿拉伯数字
% \pagenumbering{arabic}
% \setcounter{page}{1}

%%可选择这里也放一个标题
%\begin{center}
%    \title{ \Huge \textbf{{标题}}}
%\end{center}

\section{实验目的}
\begin{itemize}
  \item 深入理解C/C++程序编译、链接与构建过程
\item 提高Pthreads多线程编程综合能力
\item 并行库开发与性能验证
\end{itemize}

\section{实验内容}
\begin{itemize}
 \item 基于Pthreads的并行计算库实现:parallel\_for+调度策略
\item 动态链接库生成和使用
\item  典型问题的并行化改造:矩阵乘法,热传导问题
\end{itemize}

\section{实验过程}

\subsection{环境与工具}
简要说明实验所使用的操作系统、编译器 (gcc/g++) 版本、以及 Pthreads 库。
\begin{itemize}
    \item 编写环境:本地WSL2;运行环境:本地WSL2;gcc version 13.3.0;
    \item POSIX Threads支持: 200809
;POSIX标准版本: 200809
\end{itemize}
\subsection{核心函数实现 (\texttt{parallel\_for})}
简要描述 \texttt{parallel\_for} 函数的关键设计思路和实现。重点说明线程创建、任务划分和同步机制。
我的矩阵乘法版本的functor和热传导版本的functor在参数上不同(按道理应该尽量保持同一个形式,但是我想探索下函数指针的
用法,所以就写作了不同的形式,但其他的基本一致)。\\
此处以矩阵乘法为例:\\
我在parallel\_for函数中创建和销毁线程thread\_func,在这里执行拆解后的第一层for循环:
\begin{lstlisting}[language=C, caption={parallel\_for 函数核心代码片段}]

static void *thread_func(void *arg) {
    ThreadArgs *args = (ThreadArgs *)arg;
    int current = args->start_idx;
    //第一层for循环
    for (int t = 0; t < args->num_iters; t++) {
        //第一层for循环中要做的
        args->functor(current, args->arg);
        current += args->inc;
    }
    return NULL;
}
\end{lstlisting}
值得注意的点在于,由于每轮循环都会重新调用functor,所以functor内部的变量都是暂时的(我当时写晕了用函数内部的变量来存储local\_max\_diff)\\
下面说一下parallel\_for的划分for循环的方式,此处只考虑了for(int i = st; i < ed; i+=inc)的for循环,判定条件是
i<=ed的情况只需要在计算st与ed之间的距离时加一即可。\\
这里的划分参考了大模型给出的写法,首先上取整算出总共需要迭代的次数,然后把迭代次数分给各个线程,然后计算for循环起始位置,记录函数指针,步长等。
\begin{lstlisting}[language=C, caption={parallel\_for 函数核心代码片段}]
if (start >= end || inc <= 0 || num_threads <= 0) return;

    // 计算总迭代次数
    int total = ((end - start) + inc - 1) / inc;
    if (total <= 0) return;

    // 限制线程数不超过总迭代次数
    if (num_threads > total) num_threads = total;

    int base = total / num_threads;
    int remainder = total % num_threads;

    pthread_t *threads = malloc(num_threads * sizeof(pthread_t));
    ThreadArgs *args = malloc(num_threads * sizeof(ThreadArgs));

    for (int k = 0; k < num_threads; k++) {
        // m是迭代次数
        int m = base + (k < remainder ? 1 : 0);
        //用于计算for循环起始位置
        int sum_prev = base * k + (k < remainder ? k : remainder);
        args[k].start_idx = start + sum_prev * inc;
        args[k].num_iters = m;
        args[k].inc = inc;
        args[k].functor = functor;
        args[k].arg = arg;

        pthread_create(&threads[k], NULL, thread_func, &args[k]);
    }
\end{lstlisting}

\subsection{动态库生成与使用}
说明生成动态链接库 (.so) 的主要命令或 Makefile 规则,并简述如何在主程序 (如矩阵乘法、热传导) 中链接和调用该库。


在Makefile中,生成动态链接库(.so 文件)的核心规则如下:

\begin{lstlisting}[language=bash, basicstyle=\ttfamily\small]
$(LIBRARY): parallel_for.o
    $(CC) -shared -o $@ $^
\end{lstlisting}

\begin{itemize}
    \item \texttt{-shared}: 告诉编译器生成共享库(动态库)。
    \item \texttt{o \$@}: 输出文件名为目标名(即 \$(LIBRARY),如 libparallel\_for.so)。
    \item \texttt{\$ \^}: 依赖的所有 .o 文件(这里是 parallel\_for.o)。
\end{itemize}


主程序通过以下规则链接动态库:

\begin{lstlisting}[language=bash, basicstyle=\ttfamily\small]
$(TARGET): main.o
    $(CC) -o $@ $< -L. -lparallel_for -fopenmp -Wl,-rpath=.
\end{lstlisting}

\begin{itemize}
    \item \texttt{-L.}: 指定库文件的搜索路径为当前目录(.)。
    \item \texttt{-lparallel\_for}: 链接名为 libparallel\_for.so 的库(省略 lib 前缀和 .so 后缀)。
    \item \texttt{-Wl,-rpath=.}: 运行时动态库搜索路径为当前目录(避免 LD\_LIBRARY\_PATH 环境变量设置)。
(没有这个会找不到.so文件)
\end{itemize}
\begin{lstlisting}[language=bash, basicstyle=\ttfamily\small]
%.o: %.c
	$(CC) $(CFLAGS) -c $<
\end{lstlisting}
将所有.c结尾的文件编译为.o结尾的文件。
\subsection{应用测试 (热传导)}
简述如何将 \texttt{parallel\_for} 应用于热传导问题,替换原有的并行机制。描述测试设置,如网格大小和线程数。
\subsection{reduce的实现}
我还写了parallel\_sum和parallel\_max\_diff来模拟reduce的行为,把结果存储在线程id对应的数组位置处,最后汇总。
\begin{lstlisting}[language=c]
double parallel_sum(int start, int end, int inc, 
                   void (*functor)(int, int,void*), 
                   int num_threads, SumData *data) {
    // 分配局部和数组(第一个元素存储线程数)
    data->local_sums = calloc(num_threads, sizeof(double));
    data->thread_count = num_threads;
    
    // 执行并行计算
    parallel_for(start, end, inc, functor, data, num_threads);

    // 汇总结果
    double total = 0.0;
    for (int i = 0; i < num_threads; i++) {
        // printf("%lf ",data->local_sums[i]);
        total += data->local_sums[i];
    }
    // printf("\n");
    free(data->local_sums);
    return total;
}
\end{lstlisting}
然后其余的地方和矩阵乘法类似,定义多个结构体作为多个函数的参数,然后传递到parrallel\_for。
\section{实验结果与分析}

\subsection{性能测试结果}
展示不同线程数和调度方式下,自定义 Pthreads 实现与原始 OpenMP 实现的性能对比。
\begin{table}[h!]
\centering
\caption{矩阵乘法问题性能对比 (Pthreads vs OpenMP, 矩阵大小: 1024 x 1024)}
% Updated column definition: 4 columns total
\begin{tabular}{|c|c|c|c|}
\hline
\textbf{线程数} & \textbf{调度方式} & \textbf{自定义 Pthreads} & \textbf{原始 OpenMP} \\
\cline{3-4} % Line under Pthreads and OpenMP Time columns
 & \textbf{(Pthreads)} & 时间 (s) & 时间 (s) \\
\hline
1 (串行) & N/A & 9.6565 & 9.5752 \\
\hline
% Updated colspan to 4
\multicolumn{4}{|c|}{Pthreads: 静态调度 (Static)} \\
\hline
% Removed Speedup columns
2 & Static & 5.0383 & 5.3081 \\
4 & Static & 2.6210 & 2.4506 \\
8 & Static & 3.1723 & 3.0804 \\
16 & Static & 3.5882 & 3.3858 \\
\hline
\end{tabular}
\label{tab:performance_comparison_time_onl1y}
\end{table}
\begin{table}[h!]
\centering
\caption{热传导问题性能对比 (Pthreads vs OpenMP, 网格大小: 500 x 500, 误差0.005)}
% Updated column definition: 4 columns total
\begin{tabular}{|c|c|c|c|}
\hline
\textbf{线程数} & \textbf{调度方式} & \textbf{自定义 Pthreads} & \textbf{原始 OpenMP} \\
\cline{3-4} % Line under Pthreads and OpenMP Time columns
 & \textbf{(Pthreads)} & 时间 (s) & 时间 (s) \\
\hline
1 (串行) & N/A & 11.853832 & 9.991275 \\
\hline
% Updated colspan to 4
\multicolumn{4}{|c|}{Pthreads: 静态调度 (Static)} \\
\hline
% Removed Speedup columns
2 & Static & 6.240613 & 5.823293 \\
4 & Static & 3.641210 & 3.150620 \\
8 & Static & 5.005029 & 4.239361 \\
16 & Static & 8.068534 & 6.936898 \\
\hline
\end{tabular}
\label{tab:performance_comparison_time_only}
\end{table}

\subsection{结果分析与总结}
\subsection{矩阵乘法}
\begin{itemize}
\item 矩阵乘法的并行性能明显优于热传导问题,因为热传导需要多次创建/销毁线程,而矩阵乘法只需单次线程操作
\item 选择1024×1024矩阵尺寸既能体现并行性能,又保持合理实验时间
\item 我的Pthreads实现性能略低于OpenMP(差距约5-15%),主要由于实现细节差异,但算法核心正确
\item 所有并行结果均通过串行代码验证,在1/2/4/8/16线程下结果正确
\item 矩阵乘法展示了更好的可扩展性,在4线程时达到最佳加速比(Pthreads 3.68x,OpenMP 3.91x)
\item 当线程数超过4时,两种问题的加速比均下降,这是物理核数仅为4导致的(前面多次实验都提到了)
\end{itemize}
\begin{table}[h!]
\centering
\caption{矩阵乘法性能对比 (Pthreads vs OpenMP, 矩阵尺寸: 1024×1024)}
\resizebox{\textwidth}{!}{ % 自动调整表格宽度
\begin{tabular}{|c|c|r@{.}l|r@{.}l|r@{.}l|r@{.}l|}
\hline
\textbf{线程数} & \textbf{调度方式} & 
\multicolumn{2}{c|}{\textbf{Pthreads(s)}} & 
\multicolumn{2}{c|}{\textbf{加速比}} & 
\multicolumn{2}{c|}{\textbf{OpenMP(s)}} & 
\multicolumn{2}{c|}{\textbf{加速比}} \\
\hline
1 (串行) & N/A & 9&6565 & 1&00x & 9&5752 & 1&00x \\
\hline
\multicolumn{10}{|c|}{静态调度 (Static)} \\
\hline
2 & Static & 5&0383 & 1&92x & 5&3081 & 1&80x \\
4 & Static & 2&6210 & 3&68x & 2&4506 & 3&91x \\
8 & Static & 3&1723 & 3&04x & 3&0804 & 3&11x \\
16 & Static & 3&5882 & 2&69x & 3&3858 & 2&83x \\
\hline
\end{tabular}
}
\label{tab:matmul_performance}
\end{table}
\subsection{热传导}
\begin{itemize}
\item 最佳加速比出现在4线程时(Pthreads 2.79x,OpenMP 3.17x)
\item 相比理想线性加速(4x),实际效率仅达69.7\%(Pthreads)和79.3\%(OpenMP),这里不像矩阵乘法那样,我的程序可以和openmp的实现拥有相似的并行
性能。这是因为我的parallel\_for每次调用都伴随着线程的销毁的创建,而openmp可以先使用\#pragma omp parallel创建,然后使用\#pragma omp for调用,这里会存在很多开销(特别是在while循环里)。
这里也有解决方案,就是使用线程池,但是由于时间原因,我就只实现了基础版本。
\item 当线程数超过4时,两种问题的加速比均下降,这是物理核数仅为4导致的(前面多次实验都提到了)。我们可以发现,当线程数增多时,我的Pthread版本和Openmp版本的差距变得很明显,我觉得这也是
线程的频繁删除和创建导致的。
\end{itemize}


\begin{table}[h!]
\centering
\caption{热传导性能对比 (Pthreads vs OpenMP, 网格尺寸: 500×500)}
\resizebox{\textwidth}{!}{
\begin{tabular}{|c|c|r@{.}l|r@{.}l|r@{.}l|r@{.}l|}
\hline
\textbf{线程数} & \textbf{调度方式} & 
\multicolumn{2}{c|}{\textbf{Pthreads(s)}} & 
\multicolumn{2}{c|}{\textbf{加速比}} & 
\multicolumn{2}{c|}{\textbf{OpenMP(s)}} & 
\multicolumn{2}{c|}{\textbf{加速比}} \\
\hline
1 (串行) & N/A & 10&153832 & 1&00x & 9&991275 & 1&00x \\
\hline
\multicolumn{10}{|c|}{静态调度 (Static)} \\
\hline
2 & Static & 6&240613 & 1&63x & 5&823293 & 1&72x \\
4 & Static & 3&6412 & 2&79x & 3&1506 & 3&17x \\
8 & Static & 5&005029 & 2&03x & 4&239361 & 2&36x \\
16 & Static & 8&068534 & 1&26x & 6&936898 & 1&44x \\
\hline
\end{tabular}
}
\footnotesize 收敛误差: 0.005
\label{tab:heat_performance_final}
\end{table}

\end{document}